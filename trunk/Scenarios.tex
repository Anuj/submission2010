\section{Scenarios}
We envisioned some scenarios where people might need alternative text input mechanisms. For each scenario, we then proposed a possible mechanism that could be helpful in entering text with accuracy, and reasonable speed. It should be noted that the theme of research was not to necessarily outrun soft QWERTY in terms of speed, but to propose mechanisms that could work with reasonable performance in scenarios listed. 
\subsection{Stationary and visually focused}
This is the best case scenario, and also the one that soft-QWERTY keyboard is perfectly suited for. In this scenario, the user is stationary (not walking or moving) and can adjust to reach an optimal configuration. This includes resting the device on the lap, and entering text similar to an actual keyboard.
\subsection{Mobile and visually focused}
This is the case where the user is moving but has enough time to do operations that require visual focus on the screen. This includes scenarios like commuting (standing or sitting). The challenge here is that due to occasional loss of focus while entering the text, users have to orient themselves again after each lapse. Such scenarios would require mechanisms that don't necessarily require the used to re-orient themselves time and again. In short, mechanisms designed for such scenarios would allow the user to constantly touch the screen and still be able to signal input, as and when required. 
\subsection{Mobile and intermittently focused}
This is the worst case scenario, and it occurs in cases when the user is otherwise involved in some activity, but still has a need to enter text. This includes walking, when the user is actually focused on the path and is entering text on the side. Being able to enter text in this scenario, on devices of larger form factor is a challenge. Such scenarios would require mechanisms that have unique formations or representations that the user can memorize over time. This also means that the mechanisms would reduce the amount of visual search that the user has to perform, in order to enter a particular character. 
 
