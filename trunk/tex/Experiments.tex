\section{Study}

We designed and implemented three different mechanisms called frontside QWERTY (or QWERTY/soft-QWERTY), backside QWERTY and chording mechanism. The details of the mechanisms will be described in the next section. In this sections we describe the environment in which the mechanisms were incrementally developed and also the process that was employed. It is important to understand the structure of the study, before we try to explain the incremental changes in the design of the mechanisms.

\subsection{Participants}

For the purpose of the study we tried to keep the backgrounds of the participants mixed, but also ensured that all of them have decent exposure to typing on QWERTY keyboards (not necessarily soft-QWERTY). The participants in the study were all working professionals with more than 3 years of experience (on an average)
with QWERTY keyboards. It should be noted that all the participants in our experiment were part of a research environment and had reasonably high technical backgrounds. Moreover, we had a total of 36 participants in our study and the selection was not perfectly random. However, in spite of these limitations, working with this audience was important because our aim was to design for users who use soft-QWERTY keyboards on a regular basis. Our belief was strengthened by the finding in the post-test interviews, that out of the total of 36 participants in the study, 32 participants used soft-QWERTY keyboards on a daily basis and the rest 4 had reasonably high prior exposure, but did not own a device with soft-QWERTY keyboard. This information was important as familiarity with a particular style of text-input mechanism gets reflected in the kind of speed people achieve with the same. Though we tried to achieve a reasonable mix of backgrounds for participants, we acknowledge the fact that we still had a sample that was experienced. Therefore, our results are representative of participants who have had reasonable level of exposure to typing on QWERTY keyboards. For novices, the results might be different, in any direction.

\subsection{Phase 1: Initial test}

After we implemented the QWERTY, backside QWERTY and chording mechanism, some initial tests were conducted to get some feedback. The first one with 3 participants, and the second one with 6 participants. For the purpose of the initial tests we picked a text corpus that was different from the corpus used in the phase 2. Both the corpora were generated by choosing randomly from Scott Mackenzie's text corpus [17], and were mutually exclusive. The statistics for the two are listed in Table 1.

\begin{table}
	\centering
		\begin{tabular}{rcc}
		                         & \begin{minipage}{2cm} \centering \color{grey}{Study corpus}\end{minipage} & \begin{minipage}{2cm} \centering \color{grey}{Initial test corpus}\end{minipage}  \\ 
			 \color{grey}{Average phrase length} & 15.07 & 15.67 \\ 
			 \color{grey}{Number of words} & 76 & 81 \\ 
			 \color{grey}{Unique words} & 62 & 67 \\ 
			 \color{grey}{Min. length of word} & 1 & 1 \\ 
			 \color{grey}{Max. length of word} & 11 & 12 \\ 
			 \color{grey}{Average word length} & 4.95 & 4.43 \\ 
			 \color{grey}{Number of characters} & 437 & 425 \\ 
			 \color{grey}{Correlation with English} & 0.9297 & 0.9377 \\ 
		\end{tabular}
	\caption{Statistics for text corpora}
	\label{tab:StatisticsForTextCorpora}
\end{table}

The participants in the initial test spent 20 minutes familiarizing themselves with the interface and were then asked to enter the entire initial test corpus. The entire process was captured on videos and post-test interviews were also conducted. Based on the post-test interviews and analysis of videos, changes to the interfaces were
made to get the mechanisms ready for larger scale evaluation.

\subsection{Phase 2: Usability Study}
\subsubsection{Conditions}

The experiment had three conditions. As mentioned earlier, there were
a total of 36 participants. Each condition had 12 participants. The
three conditions were: frontside QWERTY (or QWERTY), Backside QWERTY and Chording Mechanism.

\subsubsection{Data collection methods}

To make sure that all the important aspects of the experiment and the feedback from the users is captured fully, we used multiple data collection methods:

\begin{itemize}
\item Videos: All the sessions were fully recorded. In total, around 20 hours of videos were recorded, by the end of the experiment.
\item Data Logs: All the mechanisms had a built in data logging feature, that recorded each and every action of the user, along with timestamps. This helped in understanding parts of the videos, where the user was stuck or had trouble accomplishing what they wanted.
\item Post-session interviews: After each session, the participants were asked to report on their experience with the interface. This was done to get feedback on the mechanisms and the results from these interviews are reflected in the future work section of the paper. 

\item NASA task load index: In order to quantitatively capture the
  experience with the interfaces, the NASA task load index was
  used. The NASA Task Load Index (NASA-TLX) is a subjective,
  multidimensional assessment tool that rates perceived workload on
  six different sub-scales: Mental Demand, Physical Demand, Temporal
  Demand, Performance, Effort, and Frustration. It was developed by
  the Human Performance Group at NASA's Ames Research Center over a
  three year development cycle that included more than 40 laboratory
  simulations [18]. It is a widely used method for quantifying user
  perception of task difficulty and demand on the performer of the
  task[19].
  
  It should be noted that the NASA TLX scale has 20 divisions, each
  division corresponding to 5 task load points (making it a 100 point
  scale). Researchers tend to use both the scales, depending on the
  level of granularity needed. For clarity purposes, we will always
  state the scale we are using while making an argument henceforth.

\end{itemize}
\subsubsection{Measures}
\begin{itemize}
	\item Keystrokes Per Character (KSPC): KSPC is generally treated as a measure of accuracy, because it
represents the number of keystrokes executed per character.
	\item Words Per Minute (WPM): Words Per Minute (WPM) is a measure that is commonly used to represent
the speed of a text input mechanism.
	\item Speed vs Accuracy trade-off: An efficient mechanism should achieve a trade-off between speed and accuracy. This measure studies the two measures together.
\end{itemize}
\subsubsection{Process}

The participants in the study were supposed to go through the
following steps during the study. Care was taken that the steps remain
the same across all participants so as to control the
environment. This eliminated the
possibility of the environment acting as a confounding variable in the
experiment. The process followed was:

\begin{enumerate}
\item Participants were briefed about the goal of the session. They
  were also briefed about the structure of the session.
\item They were given a brief introduction to the input
  mechanism. This was done by one of the researchers.
\item They were given the initial test corpus [Table 1] and asked to spend the next 20
  mins familiarizing themselves with the input mechanism.
\item The entire process was videotaped for data analysis and
  validation purposes.
\item After 20 minutes, they were handed over the study
  corpus. They were asked to input the corpus in its entirety, using
  the mechanism that they had just encountered. They were asked to be
  accurate with their input, and the system would underline their
  mistakes as and when they occur.
\item Once the participants had entered the entire text without any
  errors, they were handed the NASA TLX questionnaire and asked to rate
  their experience. Since the index is relative they were asked to
  compare their experience with their previous exposure to a soft
  QWERTY keyboard. 
\item Finally, they were interviewed on any other qualitative feedback
  they had on how to make the mechanisms better.
\end{enumerate}
	