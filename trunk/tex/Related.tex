\section{Related Work}

Some recent efforts have investigated alternate methods of text
inputs. Some of them like BlindType [2] (also [10]) try to let the
user type without doing visual search, and by doing ambiguity
resolution and error correction. Some other efforts like SWYPE
investigate approaches more disconnected from traditional
keypads. SWYPE [3] is a mechanism that allows users to draw spellings
on the keypad, instead of pressing individual keys. However, all of
these are for devices that are small scale and are meant for
front-side touch input.  As such, they do not address the issue of
user pose.

Wearable computing users often use one-handed chording keyboards as
their text input mechanism.  For example the Ekatetra [4], or the Twiddler [5].  One handed chording keyboards are almost
as old as computers themselves, with Doug Englebart demonstrating one
in 1968 after many years of work [6]. Chording can be
used to free up a hand for, e.g. supporting a device.  Unfortunately,
this requires a secondary device for the user, if a standard chording
keyboard is to be used.  Peripherals are problematic because they get
lost, are often stuffed away in a bag, an generally inconvenient for
spontaneous use.

With training, chording keyboards, in conjunction with phonetic
encoding of words can be used to dramatically improve the speed of
text entry.  Stenotype keyboards, used by court transcriptionists and
closed captioners, allow for transciption speeds in excess of 300
words per minute (WPM) for some (well trained) users.  For reference,
one study found that average users type at 33 WPM for transcription
and 19 WPM for composition [23].

In addition to this, researchers have also explored use of multitouch
screens for text input. Shin et al [8] implemented a multi-point touch
input mechanism and compared it against a single point
touch. Moreover, Schmidt et al [9] have investigated multitouch text
input on tabletop displays. Others like Mackenzie have tried to
analyze the effectiveness of various different mechanisms and layouts
on soft keyboards [14,15,16,17]. However, none of these systems have
tried to investigate the use of multiple touches on the back of a
device while holding the device in a more natural fashion.
