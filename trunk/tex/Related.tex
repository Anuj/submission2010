\section{Related Work}

Some recent efforts have investigated alternate methods of text inputs. Some of them like BlindType [2] (also [10]) try to let the user type without doing visual search, and by doing ambiguity resolution and error correction. Some other efforts like SWYPE investigate approaches more disconnected from traditional keypads. SWYPE [3] is a mechanism that allows users to draw spellings on the keypad, instead of pressing individual keys. However, all of these are for devices that are small scale and are meant for front-side touch input.  As such, they do not address the issue of user pose.

The Grippity keyboard \todo{cite Grippity} allows users to type on the back, using a QWERTY layout. The interface is see through, so that the users can naturally know the key that they are trying to press. However, Grippity is supposed to be a peripheral device and act as a wireless keyboard. This need not necessarily be of value in the scenarios we mentioned above. The RearType keyboard goes a step further, in providing a familiar layout of physical keys on the back of a device for text entry. This leads to less occlusion and clutter on the paired front screen. The results for RearType also look promising. \todo{cite RearType}. However, in spite of all the advantages that it offers, RearType requires additional hardware in terms of physical keys on the back of the device. Moreover, it uses the space on the back of the device to just enable text-input, whereas devices that have back-of-device touch input, use it for multiple scenarios, as in the case of LucidTouch \todo{cite LucidTouch}. LucidTouch uses touch+hover sensing to determine finger positions at the back of the device. This is then used to create the experience of semitransparency to gain greater accuracy in terms of approaching targets. Both LucidTouch and RearType require additional hardware, and are not designed to work for devices with back-of-device touch input.  The fact that there does and will continue (given the current push from the industry) to exist hardware (with back-of-device touch input) which does not include these additional hardware capability implies that techniques which do not rely on it will be needed.

Wearable computing users often use one-handed chording keyboards as their text input mechanism.  For example the Ekatetra [4], or the Twiddler [5].  One handed chording keyboards are almost as old as computers themselves, with Doug Englebart demonstrating one in 1968 after many years of work [6]. Chording can be used to free up a hand for, e.g. supporting a device.  Unfortunately, this requires a secondary device for the user, if a standard chording keyboard is to be used.  Peripherals are problematic because they get lost, are often stuffed away in a bag, an generally inconvenient for spontaneous use. However, there is prior work that suggests that chording mechanisms hold potential, sometimes even more than traditional keyboards \todo{cite Conrad}. There is also work on optimal mappings of chords to characters \todo{cite Gopher}, which can act as inspiration for future experimentation. 

With training, chording keyboards, in conjunction with phonetic encoding of words can be used to dramatically improve the speed of text entry.  Stenotype keyboards, used by court transcriptionists and closed captioners, allow for transciption speeds in excess of 300 words per minute (WPM) for some (well trained) users.  For reference, one study found that average users type at 33 WPM for transcription and 19 WPM for composition [20].

In addition to this, researchers have also explored use of multitouch screens for text input. Shin et al [8] implemented a multi-point touch input mechanism and compared it against a single point touch. Moreover, Schmidt et al [9] have investigated multitouch text input on tabletop displays. Others like Mackenzie have tried to analyze the effectiveness of various different mechanisms and layouts on soft keyboards [14]. However, none of these systems have tried to investigate the use of multiple touches on the back of a device while holding the device in a more stable posture.
