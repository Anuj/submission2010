\section{Prototype}

\subsection{Hardware implementation}

We created our hardware prototype using a Stantum Slate PC and a Stantum multitouch panel [Figure 2]. The multitouch panel was fixed on the back of the Slate PC. This replicated the same setup as a tablet with a backside touch input, and enabled us to test out the mechanisms we had implemented.

\subsection{Software architecture}

One of the major functionalities of the interfaces was projecting the touch points on the backside multitouch panel, onto Slate PC's screen. To this effect, the software architecture for all the mechanisms was split into two modules; the eventlogger and the GUI:

\subsubsection{Event-logger}

The event-logger was responsible for capturing the events being generated by the back-of-device multitouch panel. These events are then redirected to the GUI using a local socket connection. Since the volume of the events being generated was huge, modifications to the message passing routine of the event-logger were made. Instead of redirecting all the events generated on the panel, a list of cursors or touch points was maintained and was updated every 200 milliseconds. A change in the state or position of a cursor would then result in an update being sent to the GUI. This ensured that the GUI was not flooded with more updates than it can process.

\subsubsection{GUI}

The UI for the mechanisms was created in ActionScript 3.0. The design for the individual mechanisms will be explained in a following section, but on a higher level the GUI received cursor updates on the local socket. Depending on the type of update, cursor positions were updated, expired cursors were destroyed and new cursors were introduced, as and when required.
