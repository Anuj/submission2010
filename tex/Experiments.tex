\section{Experiment}
\subsection{Participants}
For the purpose of the study we tried to keep the backgrounds of the participants mixed, but also ensured that all of them have decent exposure to typing on QWERTY keyboards (not necessarily soft-QWERTY). The participants in the study were all working professionals with more than 3 years of experience (on an average) with QWERTY keyboards. According to post-test interviews, out of the total of 36 participants in the study, 32 participants had prior experience with soft-QWERTY keyboards. This information was important as familiarity with a particular style of text-input mechanism gets reflected in the kind of speed people achieve with the same. Though we tried to achieve a reasonable mix of backgrounds for participants, we acknowledge the fact that we still had a sample that was experienced. Therefore, our results are representative of participants who have had reasonable level of exposure to typing on QWERTY keyboards. For novices, the results might be different, in any direction.
\subsection{Phase 1: Usability test}
After we implemented the QWERTY, backside QWERTY and chording mechanism, we conducted two rounds of usability evaluation. The first one with 3 participants, and the second one with 6 participants. For the purpose of the usability study we picked a text corpus that was different from the corpus used in the phase 2. Both the corpora were generated by choosing randomly from Scott Mackenzie's text corpus [Reference], and were mutually exclusive. The statistics for the two are listed in Table 1.
\begin{table*}
	\centering
		\begin{tabular}{|l|c|c|} \hline
		                         & Experiment corpus & Usability test corpus \\ \hline
			 Average phrase length & 15.07 & 15.67 \\ \hline
			 Number of words & 76 & 81 \\ \hline
			 Unique words & 62 & 67 \\ \hline
			 Min. length of word & 1 & 1 \\ \hline
			 Max. length of word & 11 & 12 \\ \hline
			 Average word length & 4.95 & 4.43 \\ \hline
			 Number of characters & 437 & 425 \\ \hline
			 Correlation with English & 0.9297 & 0.9377 \\ \hline
		\end{tabular}
	\caption{Statistics for text corpora}
	\label{tab:StatisticsForTextCorpora}
\end{table*}
The participants in the usability test spent 20 minutes familiarizing themselves with the interface and were then asked to enter the entire usability test corpus. The entire process was captured on videos and post-test interviews were also conducted. Based on the post-test interviews and analysis of videos, changes to the interfaces were made.
\subsection{Phase 2: Scientific experiment}
\subsubsection{Conditions}
The experiment had three conditions. As mentioned earlier, there were a total of 36 participants. Each condition had 12 participants. The three conditions were:
\begin{itemize}
	\item QWERTY
	\item Backside QWERTY
	\item Chording Mechanism
\end{itemize}
\subsubsection{Data collection methods}
To make sure that all the important aspects of the experiment and the feedback from the users is captured fully, we used multiple data collection methods:
\begin{itemize}
	\item Videos: All the sessions were fully recorded. In total, around 20 hours of videos were recorded, by the end of the experiment. 
 	\item Data Logs: All the mechanisms had a built in data logging feature, that recorded each and every action of the user, along with timestamps. This helped in understanding parts of the videos, where the user was stuck or had trouble accomplishing what they wanted.
	\item Post-session interviews: After each session, the participants were asked to report on their experience with the interface. To give the discussion some structure, the users were asked the following questions:
\begin{enumerate}
	\item What did you like the most?
	\item What did you dislike the most?
	\item What would you change in the interface?
\end{enumerate}
	\item NASA task load index: To be able to quantitatively capture the experience with the interfaces, the NASA task load index was used. The NASA Task Load Index (NASA-TLX) is a subjective, multidimensional assessment tool that rates perceived workload on six different subscales: Mental Demand, Physical Demand, Temporal Demand, Performance, Effort, and Frustration. It was developed by the Human Performance Group at NASA's Ames Research Center over a three year development cycle that included more than 40 laboratory simulations [Reference] [Reference]. It has been cited in over 550 studies[Reference] and a recent search for NASA-TLX on Google Scholar revealed over 3,660 articles. These statistics highlight the large influence the NASA-TLX has had in Human Factors research. Therefore, we chose it as a tool in our experiment to capture the user experience. 
\end{itemize}
\subsubsection{Measures}
\begin{itemize}
	\item Keystrokes Per Character (KSPC)
	\item Words Per Minute (WPM)
	\item Speed vs Accuracy tradeoff 
\end{itemize}
\subsubsection{Process}
The participants in the study were supposed to go through the following steps during the study. Care was taken that the steps remain the same across all participants so as to control the environment. Ideally, the mechanisms should have been tested out in the scenarios that we had earlier listed. However, the lack of prior research on the topic, suggested that the first few research cycles should be conducted under controlled conditions. This eliminated the possibility of the environment acting as a confounding variable in the experiment.
\begin{enumerate}
	\item Participants were briefed about the goal of the session. They were also briefed about the structure of the session.
	\item They were given a brief introduction to the input mechanism. This was done by one of the researchers.
	\item They were given a the test corpus and asked to spend the next 20 mins familiarizing themselves with the input mechanism. The text they were supposed was the test corpus.
	\item The entire process was videotaped for data analysis and validation purposes.
	\item After 20 minutes, they were handed over the experiment corpus. They were asked to input the corpus in its entirety, using the mechanism that they had just encountered. They were asked to be accurate with their input, and the system would underline their mistakes as and when they occur.
	\item Once the participants had entered the entire text without any errors, they were handed the NASA TLX questionaire and asked to rate their experience. Since the index is relative they were asked to compare their experience with their previous exposure to a soft QWERTY keyboard. This was also done for the QWERTY mechanism, just to make sure that the mechanism is a fair representation of the soft-QWERTY family. Since the NASA-TLX is a 20 point scale, we asked them to assume that QWERTY was at 10 on each scale. This was done so that the individual ratings could be studied in details during qualitative analysis.
	\item Finally, they were interviewed on any other qualitative feedback they had on how to make the mechanisms better.
\end{enumerate}
	