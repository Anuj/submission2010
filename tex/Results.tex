\section{Results}
\subsection{Quantitative results}
We will discuss our quantitative results in terms of the three measures that we have proposed earlier in the paper. 
\subsubsection{Keystrokes Per Character (KSPC)}
KSPC is generally treated as a measure of accuracy, because it represents the number of keystrokes executed per character. From our experiment, it turned out that all the three mechanisms were similar in terms of KSPC measurements as shown in Table 2.
\begin{table*}
\begin{minipage}[b]{0.5\linewidth}
	\centering
		\begin{tabular}{|l|c|c|c|} \hline
		                         & Avg. KSPC & Max. KSPC & Min. KSPC \\ \hline
			 QWERTY & 1.288 & 1.325 & 1.239 \\ \hline
			 Chording & 1.261 & 1.373 & 1.110 \\ \hline
			 Bksd. QWERTY & 1.338 & 1.497 & 1.278 \\ \hline
		\end{tabular}
	\caption{KSPC Statistics}
	\label{tab:StatisticsForTextCorpora}
\end{minipage}	
\begin{minipage}[b]{0.5\linewidth}
	\centering
		\begin{tabular}{|l|c|c|c|} \hline
		                         & Avg. WPM & Max. WPM & Min. WPM \\ \hline
			 QWERTY & 12.162 & 15.233 & 10.2 \\ \hline
			 Chording & 4.061 & 4.79 & 2.722 \\ \hline
			 Bksd. QWERTY & 8.598 & 10.055 & 5.432 \\ \hline
		\end{tabular}
	\caption{WPM Statistics}
	\label{tab:StatisticsForTextCorpora}
\end{minipage}
\end{table*}
The Chording mechanism was on an average faster than the other two mechanisms, with the Backside QWERTY being the slowest in terms of average. However, when we did a t-test between KSPC measurements for QWERTY and chording mechanism, the p-value turned out to be 0.35. This suggests that over the test subjects there was no significant difference between the accuracies of the QWERTY and the chording mechanisms. Similarly, a t-test between KSPC measurements for QWERTY and backside QWERTY resulted in a p-value of 0.24, which still lacks significance. Therefore, both the backside touch input mechanisms were as accurate as the QWERTY mechanism. 
\subsubsection{Words Per Minute (WPM)}
Words Per Minute (WPM) is a measure that is commonly used to represent the speed of a text input mechanism. Our experiment findings suggested that QWERTY was still the fastest mechanism, followed by backside QWERTY and the slowest was chording. Statistics on WPM measurements can be found in Table 3.
The results suggested that QWERTY was faster than both backside QWERTY and chording mechanisms. However, it should be noted that the test sessions were generally around 20-30 minutes, and each user only interacted with a mechanism once. Therefore, the fact that backside QWERTY was on a average 3/4th as fast as the QWERTY mechanism was encouraging. This led us to explore the results qualitatively and also in terms of speed versus accuracy trade-off.
\subsubsection{Speed versus Accuracy Trade-off}
In spite of the discussions above, we should acknowledge that none of these measures can be studied totally independently, and there was a possibility that the participants were being more accurate by sacrificing on speed. However, since speed with a particular input mechanism is often attributed to the amount of exposure and practice, we had reasons to believe that the comparison of accuracies was still fair. However, we still did some Speed vs Accuracy analysis for the three mechanisms and [Figure] is a plot of the same.
\subsubsection{Sample tests}
Since the amount of exposure that the users received during the sessions was limited, it was obvious that lack of experience with the mechanisms is also hampering the speed and accuracy measurements. Therefore, one of the researchers who was involved in development of the interface and had reasonable exposure to the interface went through the test in exactly the same fashion as the participants. This was done to test the capability of the two new mechanisms in terms of speed and accuracy. Table 4 shows the measurements from the same.
\begin{table*}
	\centering
		\begin{tabular}{|l|c|c|c|} \hline
		                         & WPM & KSPC \\ \hline
			 Chording & 7.69 & 1.12 \\ \hline
			 Backside QWERTY & 13.231 & 1.152 \\ \hline
		\end{tabular}
	\caption{Sample Measurements}
	\label{tab:StatisticsForTextCorpora}
\end{table*}
It can be seen from the table that with decent amount of exposure to the interface, both the accuracy and the speed seem to show better trends. However, these measurements are restricted to an individual and are highly preliminary. To fully establish our claims, larger and longer studies would have to be conducted.
\subsection{Qualitative results}
As mentioned earlier, the NASA task load index was used to get subjective ratings on qualitative aspects of the interfaces. This was just done for the two new mechanisms that were implemented. This was done deliberately because the participants were asked to give the ratings keeping in mind their experience with soft-QWERTY keyboards. Since all the participants in the study had prior exposure to soft-QWERTY keyboards, this factor was uniform through the experiment. In the following few paragraphs we summarize the high-level trends that were derived from the ratings that the participants assigned to the backside QWERTY and chording mechanisms. Table [5] and [6] present the ratings given by the users to the two mechanisms.
\begin{table*}
	\centering
		\begin{tabular}{|l|c|c|c|c|c|c|c|} \hline
		                         & Mental Demand & Physical Demand & Temporal Demand & Performance & Effort & Frustration & Mean Weighted Score \\ \hline
			 User 1 & 15 & 70 & 15 & 15 & 30 & 25 & 29\\ \hline
			 User 2 & 15 & 40 & 20 & 20 & 30 & 24 & 24\\ \hline
			 User 3 & 20 & 75 & 25 & 25 & 30 & 10 & 32\\ \hline
			 User 4 & 25 & 65 & 25 & 10 & 35 & 25 & 33\\ \hline
			 User 5 & 20 & 60 & 35 & 35 & 45 & 35 & 37\\ \hline
			 User 6 & 25 & 30 & 30 & 20 & 30 & 40 & 29\\ \hline
			 User 7 & 15 & 75 & 30 & 55 & 65 & 15 & 39\\ \hline
			 User 8 & 45 & 35 & 40 & 30 & 40 & 35 & 39\\ \hline
			 User 9 & 55 & 80 & 20 & 75 & 50 & 75 & 56\\ \hline
			 User 10 & 60 & 40 & 35 & 25 & 40 & 85 & 49\\ \hline  
			 Avg. Scores & 29.5 & 57 & 27.5 & 31 & 38.5 & 37.5 & 37\\ \hline
		\end{tabular}
	\caption{NASA-TLX rating for backside-QWERTY mechanism}
	\label{tab:StatisticsForTextCorpora}
\end{table*}	
\subsubsection{Mental Demand}
By just looking at the ratings on the mental demand of the task on the NASA-TLX scale, it seemed that the users reported less mental load for the backside-QWERTY as compared to the chording mechanism. When we conducted a t-test on the ratings, it turned out to have a p-value of 0.02, which suggests that the difference was significant. It also means that the backside-QWERTY has significantly less mental load than the chording mechanism. However, the average for backside-QWERTY was 29.5 and that for chording was 43, which means that on an absolute scale participants thought that the two mechanisms were not mentally intensive to work with. The chording mechanism was deemed as harder to understand because in that case, the users were trying to work with the number of fingers as a method of input. This setup was new for all the participants in the study, and this also relates back to the low speed (in WPM) of text entry on the chording mechanism. The averages are lower than 50 for both the mechanisms, which denotes that quite a few participants believed that the mechanisms are simpler to understand and use than QWERTY. This effect was more well defined for backside-QWERTY (average of 29.5), as opposed to chording (average of 43).
\subsubsection{Physical Demand}
An analysis of the physical demand ratings on the NASA-TLX suggested that the participants in general found the chording mechanism to be as physically challenging than the backside-QWERTY. Looking at the kernel density plots of the two revealed that the distribution of population across ratings was very similar in the two interfaces. Also a t-test on the two sets of ratings resulted in a value of 0.91, which meant that the difference was not significant. Therefore, it is reasonably fair to say that the two mechanisms were equally easy or equally hard to use. However, since the averages of both the sets of ratings were around 50 (10 on the original scale) it means that none of the mechanisms were exceptionally hard to use, as opposed to each other. This also means that the two mechanisms were on an average as easy to use as the QWERTY mechanism, since the participants were assuming the QWERTY mechanism to have a rating of 10 (on the 20 point scale) for all metrics. 
\subsubsection{Temporal Demand}
This metric was important in the sense that we wanted to make sure that the participants do not feel rushed during the task. Our aim was to reproduce the natural experience of entering text, as far as possible. Therefore, the low average scores (27.5 for both mechanisms) suggested that the task was not pushing the participants to an extent that they start noticing it. There were constraints that we had specified, but none of them seemed to upset the participants. The fact that there was no time limit to the task, was helpful in this respect. 
\subsubsection{Performance}
The NASA-TLX index ratings for performance suggested that both the interfaces had good performance. The average performance ratings for both the interfaces were below 8 (on 20 point scale). It should be noted that a low rating on this scale means good performance. We also observed that there were two users who gave the chording mechanism a higher rating, thereby implying that it did not have good performance. Both these participants had writing speeds that were lass than the average. This suggests that these participants were struggling to get accustomed to the device and the mechanism. Their speed was suffering as a result of the same. We also looked at the videos from those sessions, and they corroborated the same claim.
\subsubsection{Effort}
The difference between the sets of ratings for backside-QWERTY and chording was not significant. The t-test results in a p-value of 0.52. However, studying the kernel density plots [Figure] more carefully suggested that the distribution of the opinion on the amount of effort involved in working with the backside-QWERTY was bimodal. However, for the chording it was almost evenly distributed around the average rating. The videos suggested that some of such cases were because the backside-QWERTY did not allow for change in size of the keyboard, people who had fingers longer or shorter than average finger sizes had a harder time with the mechanism as opposed to other. The chording mechanism on the other hand, did allow for such changes and therefore got an even distribution of ratings.
\begin{table*}
	\centering
		\begin{tabular}{|l|c|c|c|c|c|c|c|} \hline
		                         & Mental Demand & Physical Demand & Temporal Demand & Performance & Effort & Frustration & Mean Weighted Score \\ \hline
			 User 1 & 20 & 20 & 15 & 20 & 30 & 15 & 20\\ \hline
			 User 2 & 25 & 35 & 25 & 25 & 25 & 20 & 26\\ \hline
			 User 3 & 25 & 55 & 35 & 30 & 45 & 40 & 38\\ \hline
			 User 4 & 30 & 60 & 40 & 35 & 55 & 45 & 44\\ \hline
			 User 5 & 35 & 70 & 30 & 10 & 50 & 35 & 41\\ \hline
			 User 6 & 15 & 75 & 10 & 70 & 35 & 30 & 34\\ \hline
			 User 7 & 65 & 80 & 30 & 75 & 40 & 25 & 53\\ \hline
			 User 8 & 70 & 85 & 45 & 15 & 45 & 40 & 57\\ \hline
			 User 9 & 80 & 60 & 25 & 30 & 45 & 40 & 52\\ \hline
			 User 10 & 65 & 40 & 20 & 55 & 35 & 35 & 42\\ \hline  
			 Avg. Scores & 43 & 58 & 27.5 & 36.5 & 40.5 & 32.5 & 41\\ \hline
		\end{tabular}
	\caption{NASA-TLX rating for chording mechanism}
	\label{tab:StatisticsForTextCorpora}
\end{table*}
\subsubsection{Frustration}
The ratings suggested that on an average users were satisfied with the mechanisms. The frustration levels/ratings were restricted to the lower half of the scale for chording, however only two users reported higher levels of frustration with the backside-QWERTY. When we cross-checked this with our video logs, it turned out that both of these users had issues with getting accustomed to the mechanisms primarily because of finger sizes, as pointed out in the last section. This in turn resulted in the observed frustration on the NASA-TLX.   
\subsubsection{Mean Weighted Scores}
After the individual analysis of the ratings, we used the standardized methods to calculate the effective weighted task load ratings, as specified by NASA. Right after filling up the survey, the participants were also asked to compare metrics against each other. Since there were 6 metrics (Mental, Physical, Temporal, Performance, Effort, Frustration), there were $C_{2}^{6}$ possible combinations, and 15 questions in total. After receiving all the responses from the participants, we found that on an average effective weights for Mental Demand, Physical Demand, Temporal Demand, Performance, Effort, Frustration were 4, 3, 3, 1, 2, 2 respectively. In simple words, a higher weight means that particular dimension or metric has larger effect towards the load of the task and should be given more weightage as opposed to others. Initially we thought that interactions with backside-QWERTY and chording mechanisms were intrinsically different tasks, and therefore the weights should not be treated as the same. However, the average weights that were calculated for the two mechanisms turned out to be very close to each other. This is understandable from a viewpoint that both the tasks were actually text entry tasks with same corpus, and same kind of input method. Therefore, the value that the participants were attaching to each metric didn't change.
It is apparent from the discussion above, but to clarify again, a low weighted score on the NASA-TLX means that the overall task load was low and the experience was pleasant. The backside-QWERTY obtained a weighted score of 37 on an average, and the chording mechanism got an average score of 41. The t-test between the two resulted in a p-value value of 0.22, which means that the difference between the two sets of effective ratings was not significant. The weighted scores for both the mechanisms were low, and since the participants were comparing the mechanisms with their experiences with soft-QWERTY, it can be seen that the two new mechanisms were definitely welcome by the participants. From a qualitative standpoint, the mechanisms seemed to create good user experience, even better than a soft-QWERTY in some cases.
\subsection{Design guidelines}
After a qualitative and quantitative analysis of our results, we also did a high-level analysis of our design choices and cross-checked them against the videos. As a result of this, we came up with some major design takeaways from this piece of research. Therefore, in this section we propose some design guidelines for future efforts that look into text input by utilizing a backside touch input device. These guidelines are not sufficient, but should definitely be treated as necessary.
\subsubsection{Movement minimization}
During the study we realized that the amount of movement involved in selecting a particular character determines the speed that users would achieve with the mechanism. The post-experiment analysis of usability test videos corroborated this claim. Once we reduced the size of the keys and magnified the movement of fingers, the users could cover larger distances with smaller shifts in position. We also realized that users are able to control the finger position with very high accuracy, and therefore these optimizations help them enter text at higher speeds.
\subsubsection{Multiple finger sizes}
There can be a lot of variation in finger sizes, amongst users. We accounted for this in the chording mechanism, by having settings that user could select, if they had fingers larger or shorter than the average. This was critical for chording mechanism as the users were trying to form chords at specific locations. For backside-QWERTY, we did not make this optimization because users were not trying to position multiple fingers at the same time, and also because in that case we had tried to optimize between finger movement and key sizes. Dynamically determining the trade-off between the two, depending on the finger size would have interfered with the optimal setting of the system, and influenced other factors.
\subsubsection{Reducing dimensions of movement}
This one is only true for the chording mechanism, but it turned out from the experiment that a good way to maximize on accuracy is to reduce the number of dimensions of movement. Traditionally, with a soft-QWERTY users tend to position themselves in both, x and y co-ordinate. In the chording mechanism, the y direction was being controlled by the number of fingers, and therefore the movement was just restricted to the x direction.
\subsubsection{Visual search vs Recall}
After the usability testing, we also realized that users tend to do a visual search to find characters instead of recalling from their previous experiences with QWERTY mechanisms. It turned out the the layour of keys should be visually intuitive and familiar. As long as the positions of characters follow a pattern, either pre-existing (like QWERTY) or familiar (alphabetic), users will be able to accustom themselves in a few interactions.
\subsubsection{Pressure vs Touch}
As explained earlier, we also experimented with using pressure as a mode of input, but it turned out that it is hard for users to accurately control the amount of pressure being applied. This results in high error rates because of spurious inputs. A design fix that we used to mitigate the situation was to use touch on the front screen. Since the two thumbs are anyway used to hold the device, it was easy for the users to use them to tap on the front screen. This significantly reduced error rates from the test version to the final version.
\subsubsection{Touch Cursors}
Both our mechanisms involved showing finger positions on the screen, and we had to this in a way that we don't hide any information or don't cause a loss of perception. We achieved this by doing a number of things. We made the touch cursors transluscent, so that we don't occlude any information. We also kept the size of the cursor smaller than the size of an individual key so that they are easier to position and don't end up selecting multiple keys at the same time. 