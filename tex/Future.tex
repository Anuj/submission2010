\section{Future Work}

This research was one of the few recent efforts in the domain of
text-input mechanisms with backside touch input. There are many
directions that can serve as follow up to this work. Our
post-session/post-test interviews helped us in that regard.

Since the the speed of the chording mechanism seems to be limited by
its novelty, one important question is ``How can we quickly train
users to use the new mechanism, and maximize their engagement in the
learning process?''  As with other novel input mechanisms, if it is
too difficult for users to learn, they will not use it, even if it
performs much better ( for example, stenographer's text entry
systems).

Future research could also explore haptics and tactile feedback. This
would entail producing vibrations and other cues to signal a change on
the interface. Especially in the case of the chording mechanism,
whenever the user switches from one zone to the other, or selects a
new segment, it could be coupled with some feedback. This would also
help the interface to be used by visually challenged people as well.

The time that the users spent on the mechanisms was limited and short. They had to get accustomed to the interface in 20 minutes and then start on the test. Future research could be devoted to studying the effects of training times on KSPC and WPM. 

To further minimize the amount of movement in the chording mechanism, the future iterations of the mechanism would use the first touch to select the zone, and the rest of the touched would select segments irrespective of position. Therefore, the user won't be required to try and bring 1-3 fingers into a particular zone. Instead they would just take one finger to a particular zone and then place the others at any random location thereby selecting the segments in the zone that was selected by the first finger. This would considerably reduce the effort involved in the chording mechanism, and would also make the interface even more ready for "blind" text entry.

In the introduction we talked about how there were some scenarios that we envisioned and observed. However, since this was an exploratory study we wanted to control as many variables as possible. Therefore, very similar conditions were reproduced for all the participants. In the near future we would want to test the applications in various different scenarios, and analyze the appropriateness of different mechanisms in different scenarios.
