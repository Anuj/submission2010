\section{Conclusion}
In this paper we proposed two methods of text-input using back-of-device touch input. The methods were developed so that the problem of occlusion due to user's hands could be addressed, and users could enter text in natural postures. A study with 36 participants who had significant exposure to QWERTY keyboards was done to investigate the feasibility of such mechanisms. With less than an hour of training, users of one of the methods were able to type 3/4th as fast as soft-QWERTY. Moreover, an analysis of the perceived workload using the NASA-TLX revealed that there was no significant main effect of input technique on the perceived workload offered by the three mechanisms (soft-QWERTY, backside-QWERTY, chording). In terms of the two back-of-device input mechanisms, backside-QWERTY was found to be faster than chording (WPM), but chording was more efficient (KSPC). Moreover, our qualitative findings also suggested that the users viewed the two new mechanisms as good alternative and in general also felt that the problems of occlusion and posture were being solved. 