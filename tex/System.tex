\section{System}

\subsection{Hardware implementation}

We created a hardware prototype using a Stantum Slate PC and a Stantum
multitouch panel[Figure]. The multitouch panel was fixed on the back
of the Slate PC. This replicated the same setup as a foot-scale device
with a backside touch input, and enabled us to test out the mechanisms
we had implemented.

\subsection{Software architecture}

One of the major functionalities of the interfaces was projecting the
touch points on the backside multitouch panel, onto Slate PC's
screen. To this effect, the software architecture for all the
mechanisms was split into two modules; the eventlogger and the GUI:

\subsubsection{Eventlogger}

The eventlogger was responsible for capturing the events being
generated by the backside multitouch panel. These events were then
redirected to the GUI using a local socket connection. Since the
volume of the events being generated was huge, we had to make
modifications to the message passing routine of the
eventlogger. Instead of redirecting all the events generated on the
panel, we maintained a list of cursors or touch points and updated the
cursors every 200 milliseconds. Whenever there was a change in the
position or state of a cursor, we would send an update on the
socket. This made sure that the socket is not flooded with
information.

\subsubsection{GUI}

The UI for the mechanisms was created in actionscript 3.0. The design
for the individual mechanisms will be explained in a following
section, but on a higher level the GUI received cursor updates on the
local socket. Depending on the type of update, cursor positions were
updated, expired cursors were destroyed and new cursors were
introduced, as and when required.
