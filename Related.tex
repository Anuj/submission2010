\section{Related Work}
Back side trackpads or touch screens (e.g. a secondary touch input on the back of a device that augments the input from the front) exist on some recent devices, including the Notion Ink’s Adam tablet (http://notionink.wordpress.com/), and Samsung has applied for a patent for a method to use certain multi-touch gestures on the back of a device in collaboration with a front multi-touch panel for inputing gestures (US 20100188353).\\
Some recent efforts have also investigated alternate methods of text inputs. Some of them like BlindType (http://www.blindtype.com) try to let the user type without doing visual search, and by doing ambiguity resolution and error correction. Some other efforts like SWYPE (http://www.swypeinc.com) investigate approaches more disconnected from traditional keypads. SWYPE is a mechanism that allows users to draw spellings on the keypad, instead of pressing individual keys.\\ 
Wearable computing users often use one-handed chording keyboards as their text input mechanism.  For example the Ekatetra (http://www.ekatetra.com/), or the Twiddler (http://www.handykey.com/).  One handed chording keyboards are almost as old as computers themselves, with Doug Englebart demonstrating one in 1968 after many years of work (http://sloan.stanford.edu/mousesite/1968Demo.html).   Chording can be used to free up a hand for, e.g. supporting a device.  Unfortunately, this requires a secondary device for the user, if a standard chording keyboard is to be used.  Peripherals are problematic because they get lost, are often stuffed away in a bag, an generally inconvenient for spontaneous use.\\
With training, chording keyboards, in conjunction with phonetic encoding of words can be used to dramatically improve the speed of text entry.  Stenotype keyboards, used by court transcriptionists and closed captioners, allow for transciption speeds in excess of 300 words per minute for some (well trained) users.

